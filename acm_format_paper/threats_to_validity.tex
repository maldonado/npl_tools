\noindent\textbf{Internal validity} consider the relationship between theory and observation, in case the measured variables do not measure the actual factors. The comment patterns derived by us heavily relied on manual analysis of the code comments from Apache Ant. Like any human activity, our manual classification is subject to personal bias. To reduce this bias, any comment that was questionable was discussed between the three authors of the paper. When performing our study, we used well-commented Java projects. Since our technique heavily depends on code comments, our results and performance measures may be impacted by the quantity and quality of comments in a software project.  

When we investigate if there are refactoring recommendations to address the detected \SATD, we essentially examine if the methods in which design debt is found participate in any of the refactoring opportunities suggested by JDeodorant.
The presence of a refactoring opportunity for a given method, may not necessarily address the same kind of design debt described in the comment. In the future, we plan to investigate in a more fine-grained level the applicability of the suggested refactorings to \SATD.

When calculating the precision and recall values, we needed to manually examine the comments and label them as related to \SATD or not. Any errors in our labeling may impact the precision and recall values reported.


\noindent \textbf{External validity} consider the generalization of our findings. All of our findings were derived from comments in open source projects. To minimize external validity, we chose open source projects from different domains. That said, our results may not generalize to other open source or commercial projects. In particular, our results may not generalize to projects that have a low number or no comments.
