\todo{write compare diferent results using other classifiers}
\todo{calculate the overlap of the top 10 words of each project with the top 10 list of rq2}
In this paper we propose an approach to identify \SATD comments using Stanford Classifier. This tool, once trained correctly, can automatically classify natural language text. We create a training dataset of \SATD comments and analyzed the classification performance across ten open source projects. In RQ1, we show that our approach can outperform the current state-of-the-art in 9 out of 10 projects while identifying requirement debt. However, is not clear the reason why the approach was not as effective in this project as it was in the others.

Investigating JEdit comments we notice two main reasons 1) the project has 10,322 comments of those only 14 are requirement \SATD. The data distribution represents a to challenge the classification task. Even though the dataset is very unbalanced our precision was of 0.125 which compared with the naive baseline precision, 0.001 our approach is still more useful. 2) Most requirement \SATD comments in this project are in the middle of long comments. A lot of features identifies what is not \SATD, (i.e., due to the unbalanced dataset), and therefore long comments can generate `noise' that hinders the classification. One possible way to reduce this effect is through the addition of similar data. 

Intuitively we know that each project has its own particularities, and that each group of developers, must often, create a unique way to communicate their concerns with each other. This unique trait of source code comments is inherited from the natural language itself and render the fully automated prediction of every single \SATD very unlikely. Even analyzing a old aged project, changes in the context of the application and turnover of developers can reflect changes in the way that source code comments are written. 

For a great deal of \SATD comments there are common traits. Words as `workaround', `hack' are commonly imbued with criticism and the developers feeling that this is not the appropriate solution for the problem in hand. However, relying just in these words for the identification of \SATD is not good enough as shown in Figures \ref{fig:f1_measure_comparison_design_debt} and \ref{fig:f1_measure_comparison_requirement_debt}. Therefore, NPL techniques as proposed in our word is need in order to effectively identify \SATD comments.

In our work we set the Stanford Classifier to execute the classification using logistic regression algorithms. However would be interesting to see how other classification algorithms would perform with our dataset. Therefore, we setup Stanford Classifier to execute using two other different algorithms a Naive Bayes generative classifier and a binary classifier. 




 



% Analyzing project :apache-ant-1.7.0
% Precision:  0.0
% Recall:     0.0
% Total td found: 9
% True positives: 0
% True negatives: 3997
% False positives: 9
% False negatives: 16
% Analyzing project :apache-jmeter-2.10
% Precision:  0.0
% Recall:     0.0
% Total td found: 8
% True positives: 0
% True negatives: 7780
% False positives: 8
% False negatives: 22
% Analyzing project :argouml
% Precision:  0.0
% Recall:     0.0
% Total td found: 9
% True positives: 0
% True negatives: 8126
% False positives: 9
% False negatives: 651
% Analyzing project :columba-1.4-src
% Precision:  0.0
% Recall:     0.0
% Total td found: 2
% True positives: 0
% True negatives: 6272
% False positives: 2
% False negatives: 134
% Analyzing project :emf-2.4.1
% Precision:  0.0
% Recall:     0.0
% Total td found: 5
% True positives: 0
% True negatives: 4292
% False positives: 5
% False negatives: 16
% Analyzing project :hibernate-distribution-3.3.2.GA
% Precision:  0.0
% Recall:     0.0
% Total td found: 3
% True positives: 0
% True negatives: 2493
% False positives: 3
% False negatives: 64
% Analyzing project :jEdit-4.2
% Precision:  0.0
% Recall:     0.0
% Total td found: 6
% True positives: 0
% True negatives: 10060
% False positives: 6
% False negatives: 14
% Analyzing project :jfreechart-1.0.19
% Precision:  0.0
% Recall:     0.0
% Total td found: 1
% True positives: 0
% True negatives: 4213
% False positives: 1
% False negatives: 25
% Analyzing project :jruby-1.4.0
% Precision:  0.0
% Recall:     0.0
% Total td found: 8
% True positives: 0
% True negatives: 4267
% False positives: 8
% False negatives: 114
% Analyzing project :sql12
% Precision:  0.0
% Recall:     0.0
% Total td found: 9
% True positives: 0
% True negatives: 6935
% False positives: 9
% False negatives: 150